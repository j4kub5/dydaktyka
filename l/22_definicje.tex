% Created 2025-11-14 Fri 14:50
% Intended LaTeX compiler: pdflatex
\documentclass[11pt]{article}
\usepackage[utf8]{inputenc}
\usepackage[T1]{fontenc}
\usepackage{graphicx}
\usepackage{longtable}
\usepackage{wrapfig}
\usepackage{rotating}
\usepackage[normalem]{ulem}
\usepackage{amsmath}
\usepackage{amssymb}
\usepackage{capt-of}
\usepackage{hyperref}
\usepackage[english]{babel}
\author{Jakub Szczerbowski}
\date{\today}
\title{}
\hypersetup{
 pdfauthor={Jakub Szczerbowski},
 pdftitle={},
 pdfkeywords={},
 pdfsubject={},
 pdfcreator={Emacs 30.2 (Org mode 9.7.11)}, 
 pdflang={English}}
\begin{document}

\tableofcontents

\section{Definicje}
\label{sec:org61b820c}
\textbf{Rozdział w podręczniku:} 4

Definicja realna: wypowiedź w języku pierwszego stopnia, która charakteryzuje przedmiot i tylko ten przedmiot.

Definicja nominalna: wypowiedź w języku drugiego stopnia, które informuje o znaczeniu definiowanego słowa: \emph{Wyraz kwadrat oznacza prostokąt, który ma wszystkie boki równe.}
\subsection{Przykłady definicji (podawane przez studentów)}
\label{sec:org0c37e92}
\begin{enumerate}
\item Odcinek to jest fragment prostej, który ma początek i koniec.
\item Bursztyn to jest skamieniała żywica.
\item Wiatr to poziomy ruch powietrza z wyżu do niżu.
\item Oszustwo to jest wprowadzenie innej osoby w błąd albo wyzyskanie błędu lub niezdolności do należytego pojmowania przedsiębranego działania w celu osiągnięcia korzyści majątkowej.
\end{enumerate}
\subsection{Zadania definicji}
\label{sec:orgb406f37}
\begin{center}
\includegraphics[width=.9\linewidth]{img/zadania-definicji.png}
\label{org0fa1fff}
\end{center}


\begin{description}
\item[{Definicja sprawozdawcza}] składa sprawozdanie z tego, jak pewna grupa ludzi posługuje się wyrazem lub wyrażeniem:
\begin{itemize}
\item W języku myśliwych wyraz farba oznacza krew zwierzęcia.
\item W języku polskim wyraz czapka oznacza część garderoby noszoną na stopie.
\end{itemize}
\item[{Definicja projektująca}] ustala znaczenie jakiegoś wyrazu na przyszłość.
\begin{description}
\item[{Definicja konstrukcyjna}] ustala znaczenie na przyszłość nie licząc się z obecnym znaczeniem (tworzy nowe znaczenie, albo nowe słowo lub wyrażenie):
\begin{itemize}
\item Ilekroć w ustawie jest mowa o \textbf{przeciętnym konsumencie} - rozumie się przez to konsumenta, który jest dostatecznie dobrze poinformowany, uważny i ostrożny.
\item Dokumentem jest nośnik informacji umożliwiający zapoznanie się z jej treścią. (art. 77\textsuperscript{3} k.c.)
\end{itemize}
\item[{Definicja regulująca}] ustala znaczenie na przyszłość licząc się z obecnym znaczeniem (nazwa nieostra → nazwa ostra).
\begin{itemize}
\item Stan nietrzeźwości w rozumieniu tego kodeksu zachodzi, gdy: 1) zawartość alkoholu we krwi przekracza 0,5 promila albo prowadzi do stężenia przekraczającego tę wartość lub 2) zawartość alkoholu w 1 dm\textsuperscript{3} wydychanego powietrza przekracza 0,25 mg albo prowadzi do stężenia przekraczającego tę wartość.
\item Wysoki mężczyzna, to taki, który ma co najmniej 175 cm wzrostu.
\end{itemize}
\end{description}
\end{description}
\subsection{Budowa definicji}
\label{sec:org649fc36}
Definiendum + zwrot łączący + definiens
Bursztyn + to + kopalna żywica drzew iglastych.

\begin{itemize}
\item Definicja równościowa: \emph{definiendum + zwrot łączący + definiens}: Bursztyn to kopalna żywica drzew iglastych.
\begin{itemize}
\item Definitio per genus et differentiam specificam (definicja klasyczna):
\begin{itemize}
\item A to takie B, które ma cechę C.
\item Kwadrat to jest taki prostokąt, który ma wszystkie boki równe.
\end{itemize}
\end{itemize}
\item Definicje nierównościowe. Np. występujące w matematyce (definicja przez postulaty, definicja indukcyjna).
\end{itemize}

\begin{center}
\includegraphics[width=.9\linewidth]{img/definicje.png}
\label{orgc3822df}
\end{center}
\subsubsection{Definicje w prawie (przykłady do omówienia): art. 10 § 1 k.c., art. 627 k.c.}
\label{sec:orgc8ebd2d}

\begin{itemize}
\item Art.  10. §  1. Pełnoletnim jest, kto ukończył lat osiemnaście.
\item Art. 627. Przez umowę o dzieło przyjmujący zamówienie zobowiązuje się do wykonania oznaczonego dzieła, a zamawiający do zapłaty wynagrodzenia.
\end{itemize}
\subsection{Poprawność definicji}
\label{sec:orga0be0b7}
\begin{itemize}
\item nieprzystosowanie definicji do słownika osoby będącej adresatem definicji (ignotum per ignotum): \emph{Krącitka} to jest taka \emph{frutka}, która ma \emph{piląga}.
\item definiens zawiera definiendum (idem per idem). \emph{Polak, to jest taki człowiek, który jest narodowości polskiej.}
\item Błędne koło pośrednie: \emph{Logika to nauka o logicznym myśleniu. Logiczny to taki, który jest zgodny z nauką logiki.}
\item definicja zbyt szeroka: Człowiek to ssak dwunożny.
\item definicja zbyt wąska: Człowiek to ssak posługujący się mową i pismem.
\end{itemize}
\end{document}
